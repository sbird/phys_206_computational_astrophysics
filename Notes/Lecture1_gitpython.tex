\documentclass[12pt]{article}
 \usepackage[margin=1in]{geometry}
 \usepackage{amsmath}
 \usepackage{hyperref}

\begin{document}

\title{PHYS 206 Lecture Notes}
\author{Simeon Bird}
\date{}
\maketitle

\section{Python}

I'm going to mostly assume that you know python already.

\begin{enumerate}
 \item Use functions: ideally keep them short, 4-5 lines, and make each one have a single purpose.
 \item Add a docstring to each function. Comment the functions.
 \item Avoid global variables.
 \item Use assertions to check that your results are sane.
 \item Use the standard libraries where possible
 \item Use tests
 \item Use keyword arguments liberally for clarity
 \item Use \_ for functions not intended to be used outside, and for unused variables
 \item Use pint to automatically check your units
 \item Write tests, small self-contained functions that run your code and check it gives a consistent answer. A common python testing framework is pytest.
 \item Try to keep it simple.
\end{enumerate}

\url{https://docs.google.com/presentation/d/1Dfecu50LLvDp-XADyKf-nAgtgenAGqUk2g3kJ4TNfYQ/edit?usp=sharing}

\section{Version Control}

Version control is an automated system for keeping track of the history of your code. It allows you to avoid losing code in disasters, and it allows you to revert to known-good versions. \textbf{Use version control}.

Today's version control system is git, originally written by Linus Torvalds to solve an urgent problem. Git is a collection of small primitives strung together in versatile scripts. It is extremely powerful\footnote{A good reference is \url{https://git-scm.com/book/en/v2}, but there are several million tutorials around.}. In practice, everyone uses it through github, which has a fancy user interface.

Git provides a built-in train to move your changes from a local file to the rest of the world:

\begin{enumerate}
 \item[change$\rightarrow$] git add
 \item[staging$\rightarrow$] git commit
 \item[local commit$\rightarrow$] git push
 \item[github!]
\end{enumerate}

Each change committed to version control is stored in a list with a commit hash, like \texttt{abcde1234}. This is a unique identifier storing the state of a code.

You may use:
\begin{enumerate}
 \item git log to see the history
 \item git diff to see differences between any two commits
 \item git branch to make independent changes
 \item git pull to get remote changes for the current branch locally
\end{enumerate}

In git, branches are cheap and easy. This is the main reason git is the leading version control system. A branch is a series of changes that lead to a self-contained result. General good practice today is to do all development in a branch and only add (``merge'') it to the master branch when it is complete and (ideally) bug-free.

To create a new branch, type ``git branch newname''. To switch to it, use ``git checkout newname'' and to switch back use ``git checkout master''. To incorporate the contents of one branch into the current branch, do ``git merge newname''

\section{Github}

Github is a website frontend to git, owned by Microsoft. At the most basic, it provides a remote, searchable, copy of your code, (so that you don't forget where it is on your laptop). It also has features to facilitate collaborative development, the most important of which are:

\begin{enumerate}
 \item Pull requests: integrate and review new branches.
 \item Github actions: Automated testing of new code.
\end{enumerate}


\end{document}
