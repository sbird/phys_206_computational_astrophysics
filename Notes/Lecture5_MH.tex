\documentclass{article}
\usepackage[utf8]{inputenc}

\title{Physics 218: Fundamentals of Astrophysics}
\author{Niusha Ahvazi }
\date{October 18, 2019}

\usepackage{natbib}
\usepackage{graphicx}
\usepackage{xcolor}


\begin{document}

\maketitle

\section{MCMC Algorithm}
\subsection{Original Metropolis-Hasting}
Build a Markov chain with a steady state of the PDF.\\
\[PDF: P(x_i)\]
\[F: y(x_i)\rightarrow y(x_{i+1})\]\\
Probabilistically: \[P(x_i)	\leftrightarrow P(x_{i+1})\]\\
So you have a list of samples; which you may histogram for a PDF.\\
\\
Start with a sample, evaluate $\mathcal{L}$ at a new position chosen according to a pre-specified \underline{proposal distribution}.\\
\\
Accept or reject according to some rules.\\
\\
Once updated you select new points based on the existing samples.\\
Say we have current list of samples \( \ x_t \), want \( \ x_{t+1} \).\\
pick  \( \ x' \) at random from proposal distribution  \( \ g(x'|x_t) \)\\
compute\\
\[a=\frac{P(x')}{P(x_t)}\frac{g(x_t|x')}{g(x'|x_t)}\]\\
The new point is accepted with probability:\[x_{i+1}=x'\,\, ,\,\,p=min(1,a)  \]\\
Or rejected with \( p=1-a \), where \( x_{t+1}=x_t\).\\
\\
The trick is picking \( g(x'|x_t^*) \), the proposal.\\ ideally \(g=p\)\\
\\
Generally use a multi-Dimensional Gaussian\\
\[ g \sim exp(-(\frac{\vec{x}-\vec{\mu}}{\vec{\sigma}})^2)\]\\
\(\sigma\) is tuned to have a \(30\%\ \) acceptance fraction.\\
too low means the chain has few samples and is exploring too much.Too high means the chain has too little exploration and is stuck at a local value.\\
\\
\( \sigma \) is tuned in a "burn-in" phase, where you change it rapidly and then discard the samples to avoid a biased initial p.d.f.\\
\\
\textbf{\underline{Pros:}}\\
1)\,simple to implement\\
2)\,"Quick": new samples are fast\\
3)\,underlies other methods\\
4)\,minimal assumption: robust\\
\\
\textbf{\underline{Cons:}}\\
1)\,Stronger assumptions can have many fewer samples\\
2)\,slow for curvy degeneracy\\
3)\,Hard to parallelize\\
4)\,correlation between states\\
5)\,High-Dim. are tough\\
\\
Why high-D is hard:\\
\[ \frac{V_{sphere}}{V_{Cube}}=\frac{\frac{2r^d\pi^{\frac{d}{2}}}{d \Gamma(\frac{d}{2})}}{(2r)^d}=\frac{\pi^{\frac{d}{2}}}{d2^{d-1}\Gamma(\frac{d}{2})}\rightarrow 0\]
\end{document}
