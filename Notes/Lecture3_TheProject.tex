\documentclass[12pt]{article}

\usepackage[utf8]{inputenc}
\usepackage{latexsym,amsfonts,amssymb,amsthm,amsmath,graphicx}
\usepackage{hyperref}

\setlength{\parindent}{0in}
\setlength{\oddsidemargin}{0in}
\setlength{\textwidth}{6.5in}
\setlength{\textheight}{8.8in}
\setlength{\topmargin}{0in}
\setlength{\headheight}{18pt}

\newcommand{\Lc}{$\mathcal{L}$}


\title{ASTR 206 Lecture Notes}
\author{Simeon Bird}
\date{}
% \date{October 14, 2019}

\begin{document}

\maketitle

\section{The Project}

The main part of this course is The Project: to create a working N-body code. We have a specific science problem in mind: modelling gravitational wave events in a small black hole cluster.

The minimal code should:

\begin{enumerate}
 \item Accurately evolve the motion of particles under gravity.
 \item Have some way of visualising results.
\end{enumerate}

How will we know we have achieved goal 1?

A good way is to ensure that we correctly solve a simpler problem with an analytic solution. Thus we will add a new intermediate requirement:

\begin{enumerate}
 \item Recover the known analytic solutions for evolution of a binary.
 \item Accurately evolve the motion of particles under gravity.
 \item Have some way of visualising results.
\end{enumerate}

Possible extensions are:

\begin{enumerate}
 \item Identify objects that are about to merge by GW emission.
 \item Scale to more than a few hundred objects.
 \item Include the accretion of gas onto black holes (this one is hard!).
 \item Have a fancy 3D way of visualising results.
\end{enumerate}


\subsection{Tasks To Do}

\begin{itemize}
 \item Make project decisions.
 \item Generate initial particle positions and velocities.
 \item Compute forces between particles.
 \item Evolve particles with forces.
\end{itemize}

How should we decide which algorithms to use? Initially we have only 2 particles.



\end{document}
