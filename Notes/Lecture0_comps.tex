\documentclass[12pt]{article}
 \usepackage[margin=1in]{geometry}
 \usepackage{amsmath}
 \usepackage{hyperref}

\begin{document}

\title{PHYS 206 Lecture Notes}
\author{Simeon Bird}
\maketitle

\section{Astronomy Comprehensive Exam Policy}

Ph.D. students must pass a comprehensive exam, with two parts, a written test on the courses and an oral test on the research. They will both be taken at the end of the summer of your first year. You must pass both parts. If you do not pass, you can retake the part you didn't pass, but only once.

\subsection{Written Section of the Comprehensive Exam}

You will be examined on the following courses:
\begin{itemize}
  \item PHYS 211A – Radiative Processes in Astrophysics
  \item PHYS 213 – Astrophysics of the Interstellar Medium
  \item PHYS 215 – Dynamics \& Evolution of Galaxies
  \item PHYS 217 – Stellar Structure \& Evolution
  \item PHYS 219 – Cosmology \& Galaxy Formation
\end{itemize}

but not these courses (ie, not this one):
\begin{itemize}
  \item PHYS 214 – Techniques of Observational Astrophysics
  \item PHYS 206 – Computational Astrophysics
  \item PHYS 401 – Professional Development in Physics and Astronomy
\end{itemize}

There will be two exams of two hours each on separate days. The pass mark shall be 50\% of the total available marks. Grading will be done blind by the instructor of the course. Each course’s questions will be normalized to 20\% of the total grade.

\subsection{Research Section of the Comprehensive Exam}

You will also have to present an oral report, approximately 30 minutes in length on the background, motivation, and methods of the research study. The oral presentation will be followed by a question and answer session with a faculty committee, chaired by a comprehensive exam member. This is based on the two quarters of research time in winter and spring, and also the first summer.

The presentation should show that you have \textbf{done something} and \textbf{understood what you are doing}. Non-exhaustive examples of things you could have done include:
\begin{itemize}
  \item Written a first draft of a paper
  \item Written a proposal
  \item Done some data reduction or analysed a simulation
  \item Developed some code
\end{itemize}

To show you understand what's going on you should have read some of the significant papers (although not all of them) and talk about what problems your research addresses. It is generally a good idea to read the abstracts of the arxiv papers in your field every day.

You will get scored from $1-5$ after the exam. $\geq 3$ is a pass. $4$ is the baseline, $3$ means that you passed but we think you should do better, and $5$ means you blew us away. The committee will explain why you get the score you did.

\end{document}
