\documentclass[10pt]{article}

\usepackage[margin=1in]{geometry}
\usepackage{amsmath,amsthm,amssymb, graphicx, multicol, array, hyperref}

\newcommand{\N}{\mathbb{N}}
\newcommand{\Z}{\mathbb{Z}}

\newenvironment{problem}[2][Problem]{\begin{trivlist}
\item[\hskip \labelsep {\bfseries #1}\hskip \labelsep {\bfseries #2}]}{\end{trivlist}}

\newenvironment{segment}[2][]{\begin{trivlist}
\item[\hskip \labelsep {\bfseries #1}\hskip \labelsep {\bfseries #2}]}{\end{trivlist}}

\begin{document}

\title{Problem Set 1: Python and git}
\author{Phys 206: Computational Astrophysics}
\maketitle

\begin{problem}{1.0 Github (4)}
Create a fork of the repo here: \url{https://github.com/sbird/phys218_example}
 and clone it to your machine. Submit your answers to these problems as a github pull request to this repo.
\end{problem}

\begin{problem}{1.1 Cleanups (10)}
Edit this file: \url{https://github.com/sbird/phys218_example/blob/master/power_specs.py}
until running pylint on it produces no errors or warnings. You may ignore the warnings of the form W0621: Redefining name 'X' from outer scope (line Y) (redefined-outer-name). You may also ignore ``remarks'' and ``comments''. If code refers to a non-existent method or object you may simply delete it.

Hint: You may use search and replace to fix the whitespace issues.
\end{problem}

\begin{problem}{1.2 Find the bug (6)}
Three bugs have been introduced in:

\url{https://github.com/sbird/phys218_example/blob/master/pbhmergers.py}

Find them!
\end{problem}

\begin{problem}{1.3 Writing python (10)}

a) Write a python script, using pint, which finds the Schwarzchild radius of the Sun, in m.
%b) Using the ``halo\_mass\_function.py'' script, write a python function to compute the total number
%of halos at $z=0$ (and with the default cosmological parameters) with a mass above $10^{12} M_\odot$, by integrating  according to one of the listed halo mass function formulae.

b) Write a simple routine to multiply two matrices together in python. Write one routine using nested for loops, one using a list comprehension, and one using the built-in numpy matrix multiplication routines.

c) Write a simple test for the above routine that multiplies a large matrix with its inverse to obtain the identity. Use this test to evaluate how fast each of the routines in (b) are.
\end{problem}

Total: 36

\end{document}
