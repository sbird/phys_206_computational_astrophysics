\documentclass[10pt]{article}

\usepackage[margin=1in]{geometry}
\usepackage{amsmath,amsthm,amssymb, graphicx, multicol, array, hyperref}

\newcommand{\N}{\mathbb{N}}
\newcommand{\Z}{\mathbb{Z}}

\newenvironment{problem}[2][Problem]{\begin{trivlist}
\item[\hskip \labelsep {\bfseries #1}\hskip \labelsep {\bfseries #2}]}{\end{trivlist}}

\newenvironment{segment}[2][]{\begin{trivlist}
\item[\hskip \labelsep {\bfseries #1}\hskip \labelsep {\bfseries #2}]}{\end{trivlist}}

\begin{document}

\title{Problem Set 3: N-Body simulations}
\author{Phys 206: Computational Astrophysics}
\maketitle

\begin{problem}{3.0 Run an MP-Gadget Simulation (20)}.
a) Download and compile the MP-Gadget code for your laptop. This requires either: Mac, Linux or the Ubuntu Linux app from the windows store.

You are now going to run a small N-body simulation containing only dark matter. The example parameters in examples/dm-only can be used as a starting point for your simulation. The simulation will have $64^3$ dark matter particles (you are welcome to try more if you have a powerful computer). The size of the box should be $150$ Mpc/h. There should be no gas particles. Please also disable the radiation background and set Omega0 $= 1$ so that this is a pure matter universe.

b) Generate initial conditions.

c) Run the simulation, ideally to $z=0$, but as far as you can if your computer is slow.

d) Plot the power spectra and halo mass functions for the final redshift, and initial redshift and two intermediate redshifts. How do these evolve?

e) Use the make\_class\_power.py script to create linear theory power spectra for the same redshifts used in part d), and compare them to the power spectra output by the MP-Gadget code.
\end{problem}

\begin{problem}{3.1 Write your own N-body code (20)}.
In this problem we will write our own N-body code. You may write in python (or any other language that you prefer), and the code need use only the easiest algorithms.

a) Write the code to compute forces. Use direct summation.

b) Use the bigfile python module to load the bigfile output of the initial conditions generator from problem 3.0.

c) Implement the kick-drift-kick symplectic time integrator for time integration.

d) Compare the results to those of MP-Gadget. What are the cosmological scale factors?

e) Replace the direct summation with a tree code.
\end{problem}

%\begin{problem}{3.5 MCMC using the Gelman-Rubin criterion (6)}
%Perform the problem in Section 2.3 using $4$ independent ensembles. After $10$ samples from each ensemble, compute the Gelman-Rubin criterion explained in class. \textbf{Do not compute it using individual walkers!} How quickly does the G-R criterion converge to $<1.01$? Is it a stronger or weaker convergence criterion than the autocorrelation time?
%\end{problem}

Total: 40

\end{document}
